\documentclass[11pt]{article}

\usepackage{mathptmx}
\usepackage[hmargin=1.5in,tmargin=1in,bmargin=1.5in]{geometry}
\usepackage{graphicx}
\usepackage{multicol}
\usepackage{url}
\usepackage{xspace}
\usepackage{lettrine}
\usepackage{eso-pic}
\usepackage{picture} % explicit units in picture commands
\usepackage{tikz}

\setlength{\parindent}{0cm}
\setlength{\parskip}{0.8\baselineskip}

\begin{document}

\title{GPS Implementation Notes}
\author{Tom Sgouros}

\newcommand{\dir}[1]{\texttt{#1}}
\newcommand{\exec}[1]{\texttt{#1}}

\newcommand{\fs}{Freesurfer\xspace}

\maketitle

\section{A look at the data flow}

Roughly speaking, GPS is a processor for MEG data, with MRI data used
to help place it in context within the brain.  There is a manual for
it and a cookbook, and both of those are good references for using
it.  This document contains some implementation notes relevant to
porting the software to a new computer system.


\subsection{MRI}

GPS uses MRI data to localize signals found in the MEG data, and for
the visualization of processing results.  There is a certain amount of
processing of the raw MRI data to squeeze it into a form that is
usable by the MEG processing stream, and we describe that here.

Most of the processing of the MRI data is accomplished by a software
suite known as \fs.  This is used for segmenting the raw data
into surfaces, both the cortical surface and surfaces related to the
skull (inner, outer, skin).  The software also generates an inflated
view of the cortical surface, as well as a spherical view.

You can use \exec{freeview} to check out some of the results, which
are found in the \dir{MRI} subdirectory.  Within that directory,
you'll find subdirectories for each of the subjects, plus one for an
average subject, constructed from the other subjects in the study.
Within these, you'll find a number of subdirectories corresponding to
the various processing steps between the raw MRI and the spherical
inflation of the cortical surface.

\subsection{MEG}

The MEG data is at the heart of GPS.  The data consists of a time
series for each of the MEG detectors attached to the subject, along
with the same for each of the EEG detectors.  There are over 300 of
the former and 70 of the latter.  Before feeding it to the Granger
processing, the MEG data is aligned with the experiment's events, and
combined with the MRI data to derive a spatial location within the
brain for data using the MNE software suite.  The data records at
these spatial locations are the input to the Granger analysis.

The processed MEG data and the data byproducts from processing it with
MNE can be found in the \dir{MEG} and \dir{MNE} directories.

\subsection{Data file and directory structure}

The inputs and outputs for each step of the GPS process are contained
in a directory tree that contains five subdirectories: MRIraw, MRI,
MEG, MNE, and Granger.  The names are more or less indicative of what
is in them, but fair warning: assume nothing.

The GPS workflow is organized into four broad streams that appear as
different screens in the GUI (click on the ``Analysis'' button that
appears on the first GUI menu).

\begin{description}

\item[MRI] Processing the raw MRI data, projecting it onto a standard
  surface, building averages among the study participants.

\item[MEG] Processing the measured MEG data, and lining it up with the
  experimental ``events'' around which the study is arranged.

\item[MNE] Developing the MRI and MEG data into vertices with
  individual time series appropriate for putting together into regions
  of interest.

\item[Granger] Steps associated with the Granger causality analysis of
  the regions of interest.

\item[Utilities] There are several utility functions grouped under
  this heading, including selecting time series, and creating the
  regions of interest (ROIs).

\end{description}

The next sections will outline a rough account of the data flow among
the various subdirectories of the study directory, organized more or
less like the menu structure of the GPS/Analysis dialog.

Note that some values are cached in MAT files whose location is
specified when you initialize the system with \verb+gps_init.m+.  The
default location for these files is in the home directory of the GPS
software itself: \verb+GPS/parameters/<study name>+

\newcommand{\dataflow}[3]{\item \framebox{#1}\par%
  \textbf{Data In:~}\begin{minipage}{4.5in}\raggedright #2\end{minipage}\par%
  \textbf{Data Out:~}\begin{minipage}{4.5in}\raggedright #3\end{minipage}\par}
\renewcommand{\dir}[1]{\texttt{#1}}

\newpage

The basic data flow starts with a few ingredients:

\renewcommand{\to}{\bgroup\ttfamily$>$\egroup}

\begin{enumerate}

\item MRI data, imported with \framebox{MRI \to Import}.  (That is the
  \framebox{Import} button on the MRI dialog.)

\item MEG data, imported with \framebox{MEG \to Import}.

\item A way to group events.  This is the \framebox{MEG \to Process
    Events} stage.  This option should not be used through the GUI.

\end{enumerate}

\subsubsection{MRI Processing}

\begin{itemize}

\dataflow{Import}{Data from outside GPS, maybe Martinos DBMS
  (Bourget), maybe somewhere else on a disk.}{Data
  is put into the MRIraw folder for that subject.}
Imports the raw MRI data (DICOM files). These are high resolution
structural slices of a head, instantaneous data.  Import is Martinos
specific stuff where it's getting the DICOM data from the Martinos
DBMS (Bourget).

\dataflow{Find T1-MPRAGE}{MRI data from the \dir{MRIraw}}{Modifications to .mat
  files in \dir{GPS/parameters}, and a summary file and unpack.log in
  the \dir{MRIraw} directory} Finds the high-resolution structural MRIs
  that we are interested in -- the T1 data --- from the DICOM file.
  We are not interested in the other
  stuff, such as functional imagery, initial localizers, DTI, etc.


  Note for Mac/OSX: This part of the procedure calls the \fs function
  \verb+unpacksdcmdir+, which has a timeout in it of 20 seconds.  A
  sample dataset run on my laptop easily takes more than that to
  process, even though it's all functioning just fine.  I hacked the
  tcl script to change the timeout. (Line 983 of unpcksdcmdir.tcl)

From within Matlab on my Mac (OS X 10.12), I get this error from \verb+mri_parse_sdcmdir+:

\begin{verbatim}
mri_parse_sdcmdir --sortbyrun --d GBU_2/MRIraw/GBU_12
                              --o GBU_2/MRIraw/GBU_12/dicomdir.sumfile
                              --status GBU_2/MRIraw/GBU_12/parse.status


dyld: lazy symbol binding failed: Symbol not found: ___emutls_get_address
  Referenced from: /Applications/freesurfer/bin/../lib/gcc/lib/libgomp.1.dylib
  Expected in: /usr/lib/libSystem.B.dylib

dyld: Symbol not found: ___emutls_get_address
  Referenced from: /Applications/freesurfer/bin/../lib/gcc/lib/libgomp.1.dylib
  Expected in: /usr/lib/libSystem.B.dylib
\end{verbatim}

From outside Matlab, runs fine.  This was addressed by disabling
something called System Integrity Protection (SIP), which is some kind
of extra-OS supervision of certain system directories.  Apparently it
also governs how environments are inherited.  Specifically, it
prevents environment variables associated with dynamic libraries
(e.g. DYLD\_LIBRARY\_PATH) from being inherited when launching
protected processes.  To disable SIP, boot into the single user (hold
down cmd-R during boot), bring up a terminal window, and do
\verb+csrutil disable; reboot+.


\dataflow{Organize}{MRIraw}{MRI}
This moves the raw data into place for further processing.  For each subject,
it creates an \verb+MRI+ directory, and populates it with lightly-processed
MRI data as well as scripts, logs.  Runs \verb+mri_convert.bin+,
\verb+mri_add_xform_to_header, +\verb+mri_normalize+,
\verb+mri_em_register+, \verb+mri_watershed+, at least.  Creates and populates
subdirectories mri, scripts, and touch, and empty placeholders label, stats,
surf, tmp, and trash.

You will need a valid \fs license
to proceed past this point.  Get it from
\url{http://surfer.nmr.mgh.harvard.edu/registration.html}.

\dataflow{Build surfaces}{Data in MRI/(subject) directories}{files in label,
  mri/*.mgz, files in \dir{stats}, \dir{surf}, \dir{touch}} Invokes the \fs
\verb+recon-all+ command.  Also runs \verb+mri_em_register+,
\verb+mri_ca_register+.  This is a long one, 4 hours on my laptop.

\dataflow{FS average}{Data in \dir{MRI/surf}, others}{\dir{surf/*h.sphere} and
  \dir{surf/*h.sphere.reg}} -- copies an ``average'' subject from FS software.
%  Seppo says Nothing to do with the actual data, but that does not
%  appear to be true.
Another input.  Runs
  \verb+mris_sphere+, \verb+mris_register+, \verb+mris_make_surfaces+,
  \verb+mris_convert+, \verb+mri_volmask+,
  \verb+mris_anatomical_stats+, \verb+mri_aparc2aseg+,
  \verb+mri_binarize+, \verb+mri_segstats+,
  \verb+mri_label2label.bin+.  Also quite long.

\dataflow{Source space}{\dir{surf/*h.sphere} and
  \dir{surf/*h.sphere.reg}}{\dir{MRI/(subject)/bem}} Takes cortical
surface that \fs
  extracted, makes a grid of 3-4mm dipoles, models for the brain
  activity. Hypothesizing electrical sources at each dipole.  Still
  purely MRI processed data.  Create a tesselation of the surface and
  the nodes are the hypothesized dipoles.  Invokes
  \verb+mne_setup_source_space+.

\dataflow{Setup coreg}{\dir{MRI/(subject)/bem/*src.fif}}{\dir{MRI/(subject)/mri/brain-neuromag}} Creates
directories and prepares the ground.  The
  coregistration step is where the MRI and MEG data get combined. This
  is not that step, but is somehow getting ready for it.  Invokes
  \verb+mne_setup_mri+.

\dataflow{BE model}{\dir{MRI/(subject)/mri/T1-neuromag/sets}}%
{.surf files in \dir{MRI/(subject)/bem}, and in
  \dir{MRI/(subject)/bem/watershed}} Boundary element model.  Find
boundary to brain, differentiate with skull.  Invokes \verb+mri_watershed+

\dataflow{BE model to fif}{.surf files in \dir{MRI/(subject)/bem}, and in
  \dir{MRI/(subject)/bem/watershed}}{*-bem.fif files in \dir{MRI/(subject)/bem}}
Convert  model to fif file, maybe more.
  Invokes \verb+mne_prepare_bem_model+.

\dataflow{Brain2mat}{\dir{MRI/(subject)/surf},
  \dir{MRI/(subject)/label/*.annot},
  \dir{MNE/(subject)/*.fif}}{\dir{MRI/(subject)/brain.mat},
  \dir{MRI/avg-subject/brain-fsaverage.mat}}  Mostly rearranging.
Fills out several variables and stores them in .mat files.

\dataflow{Average surface}{files in \dir{MRI/(subject)/surf/}}%
{output in \dir{MRI/(study)\_ave}}  Process all the subjects in the
study up to the Brain2Mat step before doing this one.  This step
averages all of them so you get one average subject.  Uses
\verb+mris_make_template+,
  \verb+mris_make_average_surface+, \verb+mris_inflate+,
  \verb+mri_surf2surf+, \verb+mri_annotate+, \verb+mri_vol2vol+,
  \verb+mris_volmask+, others.

\end{itemize}

\subsubsection{MEG Preprocessing}

\begin{itemize}

\dataflow{Import}{raw fif files somewhere}{\dir{MEG/(subject)/scans}}
  The raw fif files contain a time series for each sensor on a
  subject's scalp.  This button imports a collection of raw fif files
  for a subject from somewhere specified via a GUI to a ``MEG''
  directory for each subject.  Also sets up a bunch of other
  directories: events, evoked, images, and scans\_filtered.

\dataflow{Extract events}%
{\dir{MEG/(subject)/scans}}{\dir{MEG/(subject)/events/*.eve}}
Locate the stimulus in time.  The events are
  encoded inside the raw fif file.  Extracted and stored in a .eve
  file.  Each trigger has a number.

\dataflow{Process events}{\dir{MEG/(subject)/events/*.eve}}%
{\dir{MEG/(subject)/events/*grouped.eve}}  DO NOT USE THIS.
  This triggers an experiment-specific script to scan the .eve files
  and pick up and group the events it finds in there.  Unfortunately,
  the groupings are specific to the experiment, so attempts to create
  a general solution have not succeeded.  Create your own groupings,
  and put the files in this directory when you're done so the next
  processes can be run.  The GPS Cookbook and manual have advice on
  grouping.

\dataflow{Bad channels}{text dialog}{\dir{MEG/(subject)/(subject)-bad-channels.txt}}
  This just accepts a list of bad channels from a user dialog box and records
  their numbers in a text file.  Before you do this, though, go to the `mne
  browse raw' step in the utilities menu to note which channels are
  bad, write them down, and note them here.  The channel numbers look
  like this: \verb+EEG1234+ and \verb+MEG1234+.

\item EOG projections -- go back to utilities, use `mne browse raw' to
  find eye blink events, make a fif file describing them, and specify
  the location of that file here.  Look in scans, the word is
  ``projection''  see the file 'proj.fif'.

\item Coregistration -- Go to utilities, use `mne analyze', see
  Cookbook, mapping the meg data to mri data.  Creates a cor- fif
  file, maybe in the MRI directory.  This step just identifies the
  file location and name.

\end{itemize}

\subsubsection{MNE Analysis}

\begin{itemize}

\dataflow{Average waves}{\dir{MEG/(subject)/events/*grouped*}}{\dir{MEG/(subject)/scans\_filtered}, also look in
  \dir{MNE/(subject)/blocks}, and \dir{MNE/(subject)\_ave.fif},
\dir{subject\_cov.fif}}
Creates event-related responses.  Finds data
  within the time window defined by each event.  Averaging across all
  trials for each individual subject.  look in \dir{MNE/*.ave.fif},
  all kinds of other ave  files, descriptions, the script, and so
  on. commands, blocks, higher
  level ave files, too.  This step requires you to have already specified the
  event codes belonging to each condition, a step that must be done by hand.

\dataflow{Forward solution}{\dir{MRI/(subject)/bem/*-bem.fif}}%
{\dir{MNE/(subject)/(subject)-fwd.fif}}  Requires
co-registration step from the MEG menu.

\dataflow{Inverse solution}{\dir{MNE/(subject)/(subject)-fwd.fif}}%
{\dir{MNE/(subject)/(subject)\_meg-inv.fif}
  and \dir{MNE/(subject)/(subject)\_meg\_eeg-inv.fif}} Creates *.inv.fif files.

\dataflow{Evoked trials}{\dir{MEG/(subject)/scans\_filtered/*filtered\_raw.fif}}%
{\dir{MEG/(subject)/evoked/*evoked\_filtered.mat}} Saves data
around each event in a .mat file.

\dataflow{Make .stc}{\dir{MEG/(subject)/evoked/*evoked\_filtered.mat}}%
{\dir{MNE/(subject)/stcs/*act-lh.stc} and \dir{*act-rh.stc}}
Source time code, putting MRI and MEG together.
  Tesselation of \fs is very small triangles.  So we decimate
  the grid here and impute the time series for each source point.
  Also generates some figures.

\dataflow{Morphed
  stc}{\dir{MNE/(subject)/stcs}}{\dir{MNE/(subject)/stcs/*avebrain-lh.stc}, etc}
  Transforming the individual data onto the
  spherical surface, so they can be averaged in the next step.

\dataflow{Average subject}{\dir{MNE/(subject)/stcs/*avebrain-lh.stc}}%
{\dir{MNE/(study)\_ave}} Create the average
subject from the collection of individuals.  Who all must be complete
before you get here.  That is, complete all individuals before
clicking on this button once.

\end{itemize}

\subsubsection{Granger Analysis}

\begin{itemize}

\dataflow{Process ROIs}{\dir{Granger/(study)/rois/(condition)/(AVERAGEsubject)/*.label},
\dir{MNE/(subject)/(subject)-fwd.fif}, \dir{MRI/(subject)/*.annot}}%
{\dir{Granger/rois/(condition)}, subdirectories, %
\dir{Granger/rois/(condition)/(subject)/(subject)\_rois.mat},
\dir{Granger/(study)/rois/(condition)/(subject)/*.label},}
 The ROIs for the average brain must already have been created for
 this step.  The first step here is to generate the ROIs for the
 individual brains \emph{from} the average brain, using
 \texttt{mne\_morph\_labels}. 
The label files for the average brain are created by the GPS:ROIs
  utility, in the `Utilities' dialog.  Look in
  \dir{Granger/*/rois} directory.  This step uses those ROIs to select
  comparable ROIs in the individual brains.

\dataflow{MNI Coordinates}{\dir{Granger/(subject)/rois/(condition)/*.label}}%
{\dir{Granger/rois/(study)\_ave/mni\_coordinates.txt}} For
each ROI, finds coordinates in a
  standardized brain.  % Montreal Neuro Institute.

\dataflow{ROI Timecourses}{\dir{MNE/(subject)/(subject)\_meg\_eeg-inv.fif}, %
\dir{Granger/rois/(condition)/(subject)/(subject)\_rois.mat}}%
{\dir{Granger/roiwaves/(condition)/(subject)*.mat}}
From the ROI,
pick a representative vertex.  Outputs a mat file containing a
description and the time series for each identified ROI.

\dataflow{Consolidate}{\dir{Granger/roiwaves/(condition)/(subject)*.mat}}%
{\dir{Granger/input/(study)*.mat}} The output here is a .mat file that
contains a time series for each ROI and each subject, along with
descriptions of the ROIs and the subjects.

\dataflow{Compute Granger}{\dir{Granger/input/(study)*.mat}}%
{\dir{Granger/results/raw}} Computes all Granger
possibilities.  These are the conditions that \emph{might} affect the output.

\dataflow{Null hypothesis}{\dir{Granger/results/raw}}%
{\dir{Granger/results/*.mat},
  \dir{Granger/results/nullhypothesis/*.mat}}  Cycles through the
possibilities, recalculating the error terms for the N-1 set.

\end{itemize}


\subsubsection{Utilities}

\begin{itemize}

\item FS: TkMedit

\item Visualize BEM

\dataflow{mne\_browse\_raw}{}{fif file, by default goes in
  \dir{MEG/(subject)/scans}}

\dataflow{mne\_analyze}{raw fif, also inflated
  brain}{\dir{MRI/(subject)/mri/T1-neuromag/sets}}

\dataflow{GPS: ROIs}{\dir{MNE/(subject)/stcs/*act-lh.stc}}%
{\dir{Granger/rois/(AVERAGEsubject)/*.label}}

\item GPS: Plot -- Used for making visualizations of the results.

\end{itemize}



% Under the ``MEG Preprocessing'' menu, it appears that data is imported
% from somewhere exterior to the \verb+<data>+ tree, and:

% \begin{enumerate}

% \item The ``Import'' option
% stuffs it into \verb+meg_scan_dir+ which appears to default to
% \verb+<data>/MEG/<subject>/scans+.  Unfortunately the source file
% appears to be something like \verb+/space/megraid/*/MEG/*/subj_*+.
% Don't know how to fix that.

% \item The ``Extract Events'' option seems to look for stuff in
%   \verb+<data>/MEG/<subject>/scans+ and put the output in
%   \verb+<data>/MEG/<subject>/events+

% \item There are lots of variants of the \verb+gpsa_meg_eveproc+
%   function.  Don't know what they're for.  The process events option
%   seems to want data files in
%   \verb+<data>/MEG/<subject>/events/*_*.eve+  This is the
%   \verb+meg_events_block+ option of \verb+gps_filename.m+.  The output
%   seems to be steered back to the same directory, with files called
%   \verb+*_grouped.eve+ which is the \verb+meg_events_grouped_gen+
%   option of \verb+gps_filename.m+.

% \item

% \end{enumerate}

\section{Looking at the code}

The analysis functions all seem to operate with two parameters, the
state and a flag, a `t', a `c', or a `p'. These appear to stand for  `p' for
`progress', `c' for `compute', and `t' for, well I don't know,
`test'?  The \verb+gpsa_do()+ function calls the `t' version, and that
tells it something about whether the result is subject specific or
condition specific.  Then it calls the same function with a `p', to
see whether it needs to be computed at all, and if so,
it calls the same function yet one more time with a `c'.

This allows each function to set its own conditions on whether it runs
or not, which is important with such a heterogeneous set of functions.



\end{document}
